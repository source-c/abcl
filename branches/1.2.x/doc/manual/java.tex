\paragraph{}
\label{JAVA:JAVA-EXCEPTION-CAUSE}
\index{JAVA-EXCEPTION-CAUSE}
--- Function: \textbf{java-exception-cause} [\textbf{java}] \textit{java-exception}

\begin{adjustwidth}{5em}{5em}
not-documented
\end{adjustwidth}

\paragraph{}
\label{JAVA:JCLASS-SUPERCLASS-P}
\index{JCLASS-SUPERCLASS-P}
--- Function: \textbf{jclass-superclass-p} [\textbf{java}] \textit{class-1 class-2}

\begin{adjustwidth}{5em}{5em}
Returns T if CLASS-1 is a superclass or interface of CLASS-2
\end{adjustwidth}

\paragraph{}
\label{JAVA:JINTERFACE-IMPLEMENTATION}
\index{JINTERFACE-IMPLEMENTATION}
--- Function: \textbf{jinterface-implementation} [\textbf{java}] \textit{interface \&rest method-names-and-defs}

\begin{adjustwidth}{5em}{5em}
Creates and returns an implementation of a Java interface with
   methods calling Lisp closures as given in METHOD-NAMES-AND-DEFS.

   INTERFACE is either a Java interface or a string naming one.

   METHOD-NAMES-AND-DEFS is an alternating list of method names
   (strings) and method definitions (closures).

   For missing methods, a dummy implementation is provided that
   returns nothing or null depending on whether the return type is
   void or not. This is for convenience only, and a warning is issued
   for each undefined method.
\end{adjustwidth}

\paragraph{}
\label{JAVA:DUMP-CLASSPATH}
\index{DUMP-CLASSPATH}
--- Function: \textbf{dump-classpath} [\textbf{java}] \textit{\&optional classloader}

\begin{adjustwidth}{5em}{5em}
not-documented
\end{adjustwidth}

\paragraph{}
\label{JAVA:ENSURE-JAVA-OBJECT}
\index{ENSURE-JAVA-OBJECT}
--- Function: \textbf{ensure-java-object} [\textbf{java}] \textit{obj}

\begin{adjustwidth}{5em}{5em}
Ensures OBJ is wrapped in a JAVA-OBJECT, wrapping it if necessary.
\end{adjustwidth}

\paragraph{}
\label{JAVA:JMETHOD-RETURN-TYPE}
\index{JMETHOD-RETURN-TYPE}
--- Function: \textbf{jmethod-return-type} [\textbf{java}] \textit{method}

\begin{adjustwidth}{5em}{5em}
Returns the result type (Java class) of the METHOD
\end{adjustwidth}

\paragraph{}
\label{JAVA:JFIELD-NAME}
\index{JFIELD-NAME}
--- Function: \textbf{jfield-name} [\textbf{java}] \textit{field}

\begin{adjustwidth}{5em}{5em}
Returns the name of FIELD as a Lisp string
\end{adjustwidth}

\paragraph{}
\label{JAVA:*JAVA-OBJECT-TO-STRING-LENGTH*}
\index{*JAVA-OBJECT-TO-STRING-LENGTH*}
--- Variable: \textbf{*java-object-to-string-length*} [\textbf{java}] \textit{}

\begin{adjustwidth}{5em}{5em}
Length to truncate toString() PRINT-OBJECT output for an otherwise unspecialized JAVA-OBJECT.  Can be set to NIL to indicate no limit.
\end{adjustwidth}

\paragraph{}
\label{JAVA:JINSTANCE-OF-P}
\index{JINSTANCE-OF-P}
--- Function: \textbf{jinstance-of-p} [\textbf{java}] \textit{obj class}

\begin{adjustwidth}{5em}{5em}
OBJ is an instance of CLASS (or one of its subclasses)
\end{adjustwidth}

\paragraph{}
\label{JAVA:JSTATIC-RAW}
\index{JSTATIC-RAW}
--- Function: \textbf{jstatic-raw} [\textbf{java}] \textit{method class \&rest args}

\begin{adjustwidth}{5em}{5em}
Invokes the static method METHOD on class CLASS with ARGS. Does not attempt to coerce the arguments or result into a Lisp object.
\end{adjustwidth}

\paragraph{}
\label{JAVA:DEFINE-JAVA-CLASS}
\index{DEFINE-JAVA-CLASS}
--- Macro: \textbf{define-java-class} [\textbf{java}] \textit{}

\begin{adjustwidth}{5em}{5em}
not-documented
\end{adjustwidth}

\paragraph{}
\label{JAVA:JCLASS-OF}
\index{JCLASS-OF}
--- Function: \textbf{jclass-of} [\textbf{java}] \textit{object \&optional name}

\begin{adjustwidth}{5em}{5em}
Returns the name of the Java class of OBJECT. If the NAME argument is
  supplied, verifies that OBJECT is an instance of the named class. The name
  of the class or nil is always returned as a second value.
\end{adjustwidth}

\paragraph{}
\label{JAVA:JRUN-EXCEPTION-PROTECTED}
\index{JRUN-EXCEPTION-PROTECTED}
--- Function: \textbf{jrun-exception-protected} [\textbf{java}] \textit{closure}

\begin{adjustwidth}{5em}{5em}
Invokes the function CLOSURE and returns the result.  Signals an error if stack or heap exhaustion occurs.
\end{adjustwidth}

\paragraph{}
\label{JAVA:JMETHOD-NAME}
\index{JMETHOD-NAME}
--- Function: \textbf{jmethod-name} [\textbf{java}] \textit{method}

\begin{adjustwidth}{5em}{5em}
Returns the name of METHOD as a Lisp string
\end{adjustwidth}

\paragraph{}
\label{JAVA:GET-DEFAULT-CLASSLOADER}
\index{GET-DEFAULT-CLASSLOADER}
--- Function: \textbf{get-default-classloader} [\textbf{java}] \textit{}

\begin{adjustwidth}{5em}{5em}
not-documented
\end{adjustwidth}

\paragraph{}
\label{JAVA:JCLASS-METHODS}
\index{JCLASS-METHODS}
--- Function: \textbf{jclass-methods} [\textbf{java}] \textit{class \&key declared public}

\begin{adjustwidth}{5em}{5em}
Return a vector of all (or just the declared/public, if DECLARED/PUBLIC is true) methods of CLASS
\end{adjustwidth}

\paragraph{}
\label{JAVA:GET-CURRENT-CLASSLOADER}
\index{GET-CURRENT-CLASSLOADER}
--- Function: \textbf{get-current-classloader} [\textbf{java}] \textit{}

\begin{adjustwidth}{5em}{5em}
not-documented
\end{adjustwidth}

\paragraph{}
\label{JAVA:REGISTER-JAVA-EXCEPTION}
\index{REGISTER-JAVA-EXCEPTION}
--- Function: \textbf{register-java-exception} [\textbf{java}] \textit{exception-name condition-symbol}

\begin{adjustwidth}{5em}{5em}
Registers the Java Throwable named by the symbol EXCEPTION-NAME as the condition designated by CONDITION-SYMBOL.  Returns T if successful, NIL if not.
\end{adjustwidth}

\paragraph{}
\label{JAVA:JCLASS}
\index{JCLASS}
--- Function: \textbf{jclass} [\textbf{java}] \textit{name-or-class-ref \&optional class-loader}

\begin{adjustwidth}{5em}{5em}
Returns a reference to the Java class designated by NAME-OR-CLASS-REF. If the CLASS-LOADER parameter is passed, the class is resolved with respect to the given ClassLoader.
\end{adjustwidth}

\paragraph{}
\label{JAVA:JNEW-ARRAY-FROM-LIST}
\index{JNEW-ARRAY-FROM-LIST}
--- Function: \textbf{jnew-array-from-list} [\textbf{java}] \textit{element-type list}

\begin{adjustwidth}{5em}{5em}
not-documented
\end{adjustwidth}

\paragraph{}
\label{JAVA:JMETHOD}
\index{JMETHOD}
--- Function: \textbf{jmethod} [\textbf{java}] \textit{class-ref method-name \&rest parameter-class-refs}

\begin{adjustwidth}{5em}{5em}
Returns a reference to the Java method METHOD-NAME of CLASS-REF with the given PARAMETER-CLASS-REFS.
\end{adjustwidth}

\paragraph{}
\label{JAVA:JPROPERTY-VALUE}
\index{JPROPERTY-VALUE}
--- Function: \textbf{jproperty-value} [\textbf{java}] \textit{obj prop}

\begin{adjustwidth}{5em}{5em}
not-documented
\end{adjustwidth}

\paragraph{}
\label{JAVA:JFIELD-TYPE}
\index{JFIELD-TYPE}
--- Function: \textbf{jfield-type} [\textbf{java}] \textit{field}

\begin{adjustwidth}{5em}{5em}
Returns the type (Java class) of FIELD
\end{adjustwidth}

\paragraph{}
\label{JAVA:JNEW-RUNTIME-CLASS}
\index{JNEW-RUNTIME-CLASS}
--- Function: \textbf{jnew-runtime-class} [\textbf{java}] \textit{class-name \&rest args \&key (superclass java.lang.Object) interfaces constructors methods fields (access-flags (quote (public))) annotations}

\begin{adjustwidth}{5em}{5em}
Creates and loads a Java class with methods calling Lisp closures
   as given in METHODS.  CLASS-NAME and SUPER-NAME are strings,
   INTERFACES is a list of strings, CONSTRUCTORS, METHODS and FIELDS are
   lists of constructor, method and field definitions.

   Constructor definitions - currently NOT supported - are lists of the form
   (argument-types function \&optional super-invocation-arguments)
   where argument-types is a list of strings and function is a lisp function of
   (1+ (length argument-types)) arguments; the instance (`this') is passed in as
   the last argument. The optional super-invocation-arguments is a list of numbers
   between 1 and (length argument-types), where the number k stands for the kth argument
   to the just defined constructor. If present, the constructor of the superclass
   will be called with the appropriate arguments. E.g., if the constructor definition is
   (("java.lang.String" "int") \#'(lambda (string i this) ...) (2 1))
   then the constructor of the superclass with argument types (int, java.lang.String) will
   be called with the second and first arguments.

   Method definitions are lists of the form
   (method-name return-type argument-types function \&key modifiers annotations)
   where method-name is a string, return-type and argument-types are strings or keywords for
   primitive types (:void, :int, etc.), and function is a Lisp function of minimum arity
   (1+ (length argument-types)); the instance (`this') is passed in as the first argument.

   Field definitions are lists of the form (field-name type \&key modifiers annotations).
\end{adjustwidth}

\paragraph{}
\label{JAVA:JCLASS-CONSTRUCTORS}
\index{JCLASS-CONSTRUCTORS}
--- Function: \textbf{jclass-constructors} [\textbf{java}] \textit{class}

\begin{adjustwidth}{5em}{5em}
Returns a vector of constructors for CLASS
\end{adjustwidth}

\paragraph{}
\label{JAVA:JSTATIC}
\index{JSTATIC}
--- Function: \textbf{jstatic} [\textbf{java}] \textit{method class \&rest args}

\begin{adjustwidth}{5em}{5em}
Invokes the static method METHOD on class CLASS with ARGS.
\end{adjustwidth}

\paragraph{}
\label{JAVA:JMETHOD-PARAMS}
\index{JMETHOD-PARAMS}
--- Function: \textbf{jmethod-params} [\textbf{java}] \textit{method}

\begin{adjustwidth}{5em}{5em}
Returns a vector of parameter types (Java classes) for METHOD
\end{adjustwidth}

\paragraph{}
\label{JAVA:JNEW}
\index{JNEW}
--- Function: \textbf{jnew} [\textbf{java}] \textit{constructor \&rest args}

\begin{adjustwidth}{5em}{5em}
Invokes the Java constructor CONSTRUCTOR with the arguments ARGS.
\end{adjustwidth}

\paragraph{}
\label{JAVA:JREGISTER-HANDLER}
\index{JREGISTER-HANDLER}
--- Function: \textbf{jregister-handler} [\textbf{java}] \textit{object event handler \&key data count}

\begin{adjustwidth}{5em}{5em}
not-documented
\end{adjustwidth}

\paragraph{}
\label{JAVA:JCLASS-SUPERCLASS}
\index{JCLASS-SUPERCLASS}
--- Function: \textbf{jclass-superclass} [\textbf{java}] \textit{class}

\begin{adjustwidth}{5em}{5em}
Returns the superclass of CLASS, or NIL if it hasn't got one
\end{adjustwidth}

\paragraph{}
\label{JAVA:JAVA-OBJECT-P}
\index{JAVA-OBJECT-P}
--- Function: \textbf{java-object-p} [\textbf{java}] \textit{object}

\begin{adjustwidth}{5em}{5em}
Returns T if OBJECT is a JAVA-OBJECT.
\end{adjustwidth}

\paragraph{}
\label{JAVA:JARRAY-COMPONENT-TYPE}
\index{JARRAY-COMPONENT-TYPE}
--- Function: \textbf{jarray-component-type} [\textbf{java}] \textit{atype}

\begin{adjustwidth}{5em}{5em}
Returns the component type of the array type ATYPE
\end{adjustwidth}

\paragraph{}
\label{JAVA:ADD-TO-CLASSPATH}
\index{ADD-TO-CLASSPATH}
--- Generic Function: \textbf{add-to-classpath} [\textbf{java}] \textit{}

\begin{adjustwidth}{5em}{5em}
not-documented
\end{adjustwidth}

\paragraph{}
\label{JAVA:UNREGISTER-JAVA-EXCEPTION}
\index{UNREGISTER-JAVA-EXCEPTION}
--- Function: \textbf{unregister-java-exception} [\textbf{java}] \textit{exception-name}

\begin{adjustwidth}{5em}{5em}
Unregisters the Java Throwable EXCEPTION-NAME previously registered by REGISTER-JAVA-EXCEPTION.
\end{adjustwidth}

\paragraph{}
\label{JAVA:JOBJECT-LISP-VALUE}
\index{JOBJECT-LISP-VALUE}
--- Function: \textbf{jobject-lisp-value} [\textbf{java}] \textit{java-object}

\begin{adjustwidth}{5em}{5em}
Attempts to coerce JAVA-OBJECT into a Lisp object.
\end{adjustwidth}

\paragraph{}
\label{JAVA:JCLASS-NAME}
\index{JCLASS-NAME}
--- Function: \textbf{jclass-name} [\textbf{java}] \textit{class-ref \&optional name}

\begin{adjustwidth}{5em}{5em}
When called with one argument, returns the name of the Java class
  designated by CLASS-REF. When called with two arguments, tests
  whether CLASS-REF matches NAME.
\end{adjustwidth}

\paragraph{}
\label{JAVA:JARRAY-FROM-LIST}
\index{JARRAY-FROM-LIST}
--- Function: \textbf{jarray-from-list} [\textbf{java}] \textit{list}

\begin{adjustwidth}{5em}{5em}
Return a Java array from LIST whose type is inferred from the first element.

For more control over the type of the array, use JNEW-ARRAY-FROM-LIST.
\end{adjustwidth}

\paragraph{}
\label{JAVA:JMEMBER-PUBLIC-P}
\index{JMEMBER-PUBLIC-P}
--- Function: \textbf{jmember-public-p} [\textbf{java}] \textit{member}

\begin{adjustwidth}{5em}{5em}
MEMBER is a public member of its declaring class
\end{adjustwidth}

\paragraph{}
\label{JAVA:+NULL+}
\index{+NULL+}
--- Variable: \textbf{+null+} [\textbf{java}] \textit{}

\begin{adjustwidth}{5em}{5em}
The JVM null object reference.
\end{adjustwidth}

\paragraph{}
\label{JAVA:ENSURE-JAVA-CLASS}
\index{ENSURE-JAVA-CLASS}
--- Function: \textbf{ensure-java-class} [\textbf{java}] \textit{jclass}

\begin{adjustwidth}{5em}{5em}
not-documented
\end{adjustwidth}

\paragraph{}
\label{JAVA:JAVA-CLASS}
\index{JAVA-CLASS}
--- Class: \textbf{java-class} [\textbf{java}] \textit{}

\begin{adjustwidth}{5em}{5em}
not-documented
\end{adjustwidth}

\paragraph{}
\label{JAVA:JMETHOD-LET}
\index{JMETHOD-LET}
--- Macro: \textbf{jmethod-let} [\textbf{java}] \textit{}

\begin{adjustwidth}{5em}{5em}
not-documented
\end{adjustwidth}

\paragraph{}
\label{JAVA:JCLASS-ARRAY-P}
\index{JCLASS-ARRAY-P}
--- Function: \textbf{jclass-array-p} [\textbf{java}] \textit{class}

\begin{adjustwidth}{5em}{5em}
Returns T if CLASS is an array class
\end{adjustwidth}

\paragraph{}
\label{JAVA:JCALL}
\index{JCALL}
--- Function: \textbf{jcall} [\textbf{java}] \textit{method-ref instance \&rest args}

\begin{adjustwidth}{5em}{5em}
Invokes the Java method METHOD-REF on INSTANCE with arguments ARGS, coercing the result into a Lisp object, if possible.
\end{adjustwidth}

\paragraph{}
\label{JAVA:JARRAY-REF-RAW}
\index{JARRAY-REF-RAW}
--- Function: \textbf{jarray-ref-raw} [\textbf{java}] \textit{java-array \&rest indices}

\begin{adjustwidth}{5em}{5em}
Dereference the Java array JAVA-ARRAY using the given INDICIES. Does not attempt to coerce the result into a Lisp object.
\end{adjustwidth}

\paragraph{}
\label{JAVA:JEQUAL}
\index{JEQUAL}
--- Function: \textbf{jequal} [\textbf{java}] \textit{obj1 obj2}

\begin{adjustwidth}{5em}{5em}
Compares obj1 with obj2 using java.lang.Object.equals()
\end{adjustwidth}

\paragraph{}
\label{JAVA:JNULL-REF-P}
\index{JNULL-REF-P}
--- Function: \textbf{jnull-ref-p} [\textbf{java}] \textit{object}

\begin{adjustwidth}{5em}{5em}
Returns a non-NIL value when the JAVA-OBJECT `object` is `null`,
or signals a TYPE-ERROR condition if the object isn't of
the right type.
\end{adjustwidth}

\paragraph{}
\label{JAVA:JNEW-ARRAY}
\index{JNEW-ARRAY}
--- Function: \textbf{jnew-array} [\textbf{java}] \textit{element-type \&rest dimensions}

\begin{adjustwidth}{5em}{5em}
Creates a new Java array of type ELEMENT-TYPE, with the given DIMENSIONS.
\end{adjustwidth}

\paragraph{}
\label{JAVA:CHAIN}
\index{CHAIN}
--- Macro: \textbf{chain} [\textbf{java}] \textit{}

\begin{adjustwidth}{5em}{5em}
not-documented
\end{adjustwidth}

\paragraph{}
\label{JAVA:JFIELD}
\index{JFIELD}
--- Function: \textbf{jfield} [\textbf{java}] \textit{class-ref-or-field field-or-instance \&optional instance value}

\begin{adjustwidth}{5em}{5em}
Retrieves or modifies a field in a Java class or instance.

Supported argument patterns:

   Case 1: class-ref  field-name:
      Retrieves the value of a static field.

   Case 2: class-ref  field-name  instance-ref:
      Retrieves the value of a class field of the instance.

   Case 3: class-ref  field-name  primitive-value:
      Stores a primitive-value in a static field.

   Case 4: class-ref  field-name  instance-ref  value:
      Stores value in a class field of the instance.

   Case 5: class-ref  field-name  nil  value:
      Stores value in a static field (when value may be
      confused with an instance-ref).

   Case 6: field-name  instance:
      Retrieves the value of a field of the instance. The
      class is derived from the instance.

   Case 7: field-name  instance  value:
      Stores value in a field of the instance. The class is
      derived from the instance.


\end{adjustwidth}

\paragraph{}
\label{JAVA:JAVA-OBJECT}
\index{JAVA-OBJECT}
--- Class: \textbf{java-object} [\textbf{java}] \textit{}

\begin{adjustwidth}{5em}{5em}
not-documented
\end{adjustwidth}

\paragraph{}
\label{JAVA:JCLASS-INTERFACES}
\index{JCLASS-INTERFACES}
--- Function: \textbf{jclass-interfaces} [\textbf{java}] \textit{class}

\begin{adjustwidth}{5em}{5em}
Returns the vector of interfaces of CLASS
\end{adjustwidth}

\paragraph{}
\label{JAVA:+TRUE+}
\index{+TRUE+}
--- Variable: \textbf{+true+} [\textbf{java}] \textit{}

\begin{adjustwidth}{5em}{5em}
The JVM primitive value for boolean true.
\end{adjustwidth}

\paragraph{}
\label{JAVA:JMAKE-INVOCATION-HANDLER}
\index{JMAKE-INVOCATION-HANDLER}
--- Function: \textbf{jmake-invocation-handler} [\textbf{java}] \textit{function}

\begin{adjustwidth}{5em}{5em}
not-documented
\end{adjustwidth}

\paragraph{}
\label{JAVA:JRESOLVE-METHOD}
\index{JRESOLVE-METHOD}
--- Function: \textbf{jresolve-method} [\textbf{java}] \textit{method-name instance \&rest args}

\begin{adjustwidth}{5em}{5em}
Finds the most specific Java method METHOD-NAME on INSTANCE applicable to arguments ARGS. Returns NIL if no suitable method is found. The algorithm used for resolution is the same used by JCALL when it is called with a string as the first parameter (METHOD-REF).
\end{adjustwidth}

\paragraph{}
\label{JAVA:MAKE-CLASSLOADER}
\index{MAKE-CLASSLOADER}
--- Function: \textbf{make-classloader} [\textbf{java}] \textit{\&optional parent}

\begin{adjustwidth}{5em}{5em}
not-documented
\end{adjustwidth}

\paragraph{}
\label{JAVA:JMEMBER-PROTECTED-P}
\index{JMEMBER-PROTECTED-P}
--- Function: \textbf{jmember-protected-p} [\textbf{java}] \textit{member}

\begin{adjustwidth}{5em}{5em}
MEMBER is a protected member of its declaring class
\end{adjustwidth}

\paragraph{}
\label{JAVA:MAKE-IMMEDIATE-OBJECT}
\index{MAKE-IMMEDIATE-OBJECT}
--- Function: \textbf{make-immediate-object} [\textbf{java}] \textit{object \&optional type}

\begin{adjustwidth}{5em}{5em}
Attempts to coerce a given Lisp object into a java-object of the
given type.  If type is not provided, works as jobject-lisp-value.
Currently, type may be :BOOLEAN, treating the object as a truth value,
or :REF, which returns Java null if NIL is provided.

Deprecated.  Please use JAVA:+NULL+, JAVA:+TRUE+, and JAVA:+FALSE+ for
constructing wrapped primitive types, JAVA:JOBJECT-LISP-VALUE for converting a
JAVA:JAVA-OBJECT to a Lisp value, or JAVA:JNULL-REF-P to distinguish a wrapped
null JAVA-OBJECT from NIL.
\end{adjustwidth}

\paragraph{}
\label{JAVA:JNEW-ARRAY-FROM-ARRAY}
\index{JNEW-ARRAY-FROM-ARRAY}
--- Function: \textbf{jnew-array-from-array} [\textbf{java}] \textit{element-type array}

\begin{adjustwidth}{5em}{5em}
Returns a new Java array with base type ELEMENT-TYPE (a string or a class-ref)
   initialized from ARRAY
\end{adjustwidth}

\paragraph{}
\label{JAVA:JOBJECT-CLASS}
\index{JOBJECT-CLASS}
--- Function: \textbf{jobject-class} [\textbf{java}] \textit{obj}

\begin{adjustwidth}{5em}{5em}
Returns the Java class that OBJ belongs to
\end{adjustwidth}

\paragraph{}
\label{JAVA:JCLASS-FIELDS}
\index{JCLASS-FIELDS}
--- Function: \textbf{jclass-fields} [\textbf{java}] \textit{class \&key declared public}

\begin{adjustwidth}{5em}{5em}
Returns a vector of all (or just the declared/public, if DECLARED/PUBLIC is true) fields of CLASS
\end{adjustwidth}

\paragraph{}
\label{JAVA:JAVA-EXCEPTION}
\index{JAVA-EXCEPTION}
--- Class: \textbf{java-exception} [\textbf{java}] \textit{}

\begin{adjustwidth}{5em}{5em}
not-documented
\end{adjustwidth}

\paragraph{}
\label{JAVA:DESCRIBE-JAVA-OBJECT}
\index{DESCRIBE-JAVA-OBJECT}
--- Function: \textbf{describe-java-object} [\textbf{java}] \textit{}

\begin{adjustwidth}{5em}{5em}
not-documented
\end{adjustwidth}

\paragraph{}
\label{JAVA:JFIELD-RAW}
\index{JFIELD-RAW}
--- Function: \textbf{jfield-raw} [\textbf{java}] \textit{class-ref-or-field field-or-instance \&optional instance value}

\begin{adjustwidth}{5em}{5em}
Retrieves or modifies a field in a Java class or instance. Does not
attempt to coerce its value or the result into a Lisp object.

Supported argument patterns:

   Case 1: class-ref  field-name:
      Retrieves the value of a static field.

   Case 2: class-ref  field-name  instance-ref:
      Retrieves the value of a class field of the instance.

   Case 3: class-ref  field-name  primitive-value:
      Stores a primitive-value in a static field.

   Case 4: class-ref  field-name  instance-ref  value:
      Stores value in a class field of the instance.

   Case 5: class-ref  field-name  nil  value:
      Stores value in a static field (when value may be
      confused with an instance-ref).

   Case 6: field-name  instance:
      Retrieves the value of a field of the instance. The
      class is derived from the instance.

   Case 7: field-name  instance  value:
      Stores value in a field of the instance. The class is
      derived from the instance.


\end{adjustwidth}

\paragraph{}
\label{JAVA:JCONSTRUCTOR-PARAMS}
\index{JCONSTRUCTOR-PARAMS}
--- Function: \textbf{jconstructor-params} [\textbf{java}] \textit{constructor}

\begin{adjustwidth}{5em}{5em}
Returns a vector of parameter types (Java classes) for CONSTRUCTOR
\end{adjustwidth}

\paragraph{}
\label{JAVA:JMEMBER-STATIC-P}
\index{JMEMBER-STATIC-P}
--- Function: \textbf{jmember-static-p} [\textbf{java}] \textit{member}

\begin{adjustwidth}{5em}{5em}
MEMBER is a static member of its declaring class
\end{adjustwidth}

\paragraph{}
\label{JAVA:JCOERCE}
\index{JCOERCE}
--- Function: \textbf{jcoerce} [\textbf{java}] \textit{object intended-class}

\begin{adjustwidth}{5em}{5em}
Attempts to coerce OBJECT into a JavaObject of class INTENDED-CLASS.  Raises a TYPE-ERROR if no conversion is possible.
\end{adjustwidth}

\paragraph{}
\label{JAVA:JCONSTRUCTOR}
\index{JCONSTRUCTOR}
--- Function: \textbf{jconstructor} [\textbf{java}] \textit{class-ref \&rest parameter-class-refs}

\begin{adjustwidth}{5em}{5em}
Returns a reference to the Java constructor of CLASS-REF with the given PARAMETER-CLASS-REFS.
\end{adjustwidth}

\paragraph{}
\label{JAVA:JARRAY-SET}
\index{JARRAY-SET}
--- Function: \textbf{jarray-set} [\textbf{java}] \textit{java-array new-value \&rest indices}

\begin{adjustwidth}{5em}{5em}
Stores NEW-VALUE at the given index in JAVA-ARRAY.
\end{adjustwidth}

\paragraph{}
\label{JAVA:JARRAY-LENGTH}
\index{JARRAY-LENGTH}
--- Function: \textbf{jarray-length} [\textbf{java}] \textit{java-array}

\begin{adjustwidth}{5em}{5em}
not-documented
\end{adjustwidth}

\paragraph{}
\label{JAVA:JARRAY-REF}
\index{JARRAY-REF}
--- Function: \textbf{jarray-ref} [\textbf{java}] \textit{java-array \&rest indices}

\begin{adjustwidth}{5em}{5em}
Dereferences the Java array JAVA-ARRAY using the given INDICIES, coercing the result into a Lisp object, if possible.
\end{adjustwidth}

\paragraph{}
\label{JAVA:JCLASS-FIELD}
\index{JCLASS-FIELD}
--- Function: \textbf{jclass-field} [\textbf{java}] \textit{class field-name}

\begin{adjustwidth}{5em}{5em}
Returns the field named FIELD-NAME of CLASS
\end{adjustwidth}

\paragraph{}
\label{JAVA:JMAKE-PROXY}
\index{JMAKE-PROXY}
--- Generic Function: \textbf{jmake-proxy} [\textbf{java}] \textit{}

\begin{adjustwidth}{5em}{5em}
not-documented
\end{adjustwidth}

\paragraph{}
\label{JAVA:JCALL-RAW}
\index{JCALL-RAW}
--- Function: \textbf{jcall-raw} [\textbf{java}] \textit{method-ref instance \&rest args}

\begin{adjustwidth}{5em}{5em}
Invokes the Java method METHOD-REF on INSTANCE with arguments ARGS. Does not attempt to coerce the result into a Lisp object.
\end{adjustwidth}

\paragraph{}
\label{JAVA:+FALSE+}
\index{+FALSE+}
--- Variable: \textbf{+false+} [\textbf{java}] \textit{}

\begin{adjustwidth}{5em}{5em}
The JVM primitive value for boolean false.
\end{adjustwidth}

\paragraph{}
\label{JAVA:JCLASS-INTERFACE-P}
\index{JCLASS-INTERFACE-P}
--- Function: \textbf{jclass-interface-p} [\textbf{java}] \textit{class}

\begin{adjustwidth}{5em}{5em}
Returns T if CLASS is an interface
\end{adjustwidth}

