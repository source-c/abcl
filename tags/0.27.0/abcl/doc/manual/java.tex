\begin{verbatim}

%JGET-PROPERTY-VALUE
  Function: Gets a JavaBeans property on JAVA-OBJECT.

%JSET-PROPERTY-VALUE
  Function: Sets a JavaBean property on JAVA-OBJECT.

*JAVA-OBJECT-TO-STRING-LENGTH* 
 Variable: Length to truncate toString()
 PRINT-OBJECT output for an otherwise unspecialized JAVA-OBJECT.  Can
 be set to NIL to indicate no limit.

ADD-TO-CLASSPATH
  Function: (not documented)

CHAIN
  Function: (not documented)

DESCRIBE-JAVA-OBJECT
  Function: (not documented)

DUMP-CLASSPATH
  Function: (not documented)

ENSURE-JAVA-CLASS
  Function: (not documented)

ENSURE-JAVA-OBJECT
  Function: Ensures OBJ is wrapped in a JAVA-OBJECT, wrapping it if necessary.

GET-DEFAULT-CLASSLOADER
  Function: (not documented)

JARRAY-COMPONENT-TYPE
  Function: Returns the component type of the array type ATYPE

JARRAY-LENGTH
  Function: (not documented)

JARRAY-REF
 Function: Dereferences the Java array JAVA-ARRAY using the given
 INDICIES, coercing the result into a Lisp object, if possible.

JARRAY-REF-RAW
 Function: Dereference the Java array JAVA-ARRAY using the given
 INDICIES. Does not attempt to coerce the result into a Lisp object.

JARRAY-SET
  Function: Stores NEW-VALUE at the given index in JAVA-ARRAY.

JAVA-CLASS
  Class: (not documented)

JAVA-EXCEPTION
  Class: (not documented)

JAVA-EXCEPTION-CAUSE
  Function: Returns the cause of JAVA-EXCEPTION. (The cause is the Java Throwable

JAVA-OBJECT
  Class: (not documented)

JAVA-OBJECT-P
  Function: Returns T if OBJECT is a JAVA-OBJECT.

JCALL
 Function: Invokes the Java method METHOD-REF on INSTANCE with
 arguments ARGS, coercing the result into a Lisp object, if possible.

JCALL-RAW
 Function: Invokes the Java method METHOD-REF on INSTANCE with
 arguments ARGS. Does not attempt to coerce the result into a Lisp
 object.

JCLASS
 Function: Returns a reference to the Java class designated by
 NAME-OR-CLASS-REF. If the CLASS-LOADER parameter is passed, the class
 is resolved with respect to the given ClassLoader.

JCLASS-ARRAY-P
  Function: Returns T if CLASS is an array class

JCLASS-CONSTRUCTORS
  Function: Returns a vector of constructors for CLASS

JCLASS-FIELD
  Function: Returns the field named FIELD-NAME of CLASS

JCLASS-FIELDS
  Function: Returns a vector of all (or just the declared/public, if
  DECLARED/PUBLIC is true) fields of CLASS

JCLASS-INTERFACE-P
  Function: Returns T if CLASS is an interface

JCLASS-INTERFACES
  Function: Returns the vector of interfaces of CLASS

JCLASS-METHODS
  Function: Return a vector of all (or just the declared/public, if
 DECLARED/PUBLIC is true) methods of CLASS

JCLASS-NAME
  Function: When called with one argument, returns the name of the Java class

JCLASS-OF
  Function: (not documented)

JCLASS-SUPERCLASS
  Function: Returns the superclass of CLASS, or NIL if it hasn't got one

JCLASS-SUPERCLASS-P
  Function: Returns T if CLASS-1 is a superclass or interface of CLASS-2

JCOERCE
  Function: Attempts to coerce OBJECT into a JavaObject of class
 INTENDED-CLASS.  Raises a TYPE-ERROR if no conversion is possible.

JCONSTRUCTOR
  Function: Returns a reference to the Java constructor of CLASS-REF
  with the given PARAMETER-CLASS-REFS.

JCONSTRUCTOR-PARAMS
  Function: Returns a vector of parameter types (Java classes) for CONSTRUCTOR

JEQUAL
  Function: Compares obj1 with obj2 using java.lang.Object.equals()

JFIELD
  Function: Retrieves or modifies a field in a Java class or instance.

JFIELD-NAME
  Function: Returns the name of FIELD as a Lisp string

JFIELD-RAW
  Function: Retrieves or modifies a field in a Java class or instance. Does not

JFIELD-TYPE
  Function: Returns the type (Java class) of FIELD

JINSTANCE-OF-P
  Function: OBJ is an instance of CLASS (or one of its subclasses)

JINTERFACE-IMPLEMENTATION
  Function: Creates and returns an implementation of a Java interface with

JMAKE-INVOCATION-HANDLER
  Function: (not documented)

JMAKE-PROXY
  Function: (not documented)

JMEMBER-PROTECTED-P
  Function: MEMBER is a protected member of its declaring class

JMEMBER-PUBLIC-P
  Function: MEMBER is a public member of its declaring class

JMEMBER-STATIC-P
  Function: MEMBER is a static member of its declaring class

JMETHOD
  Function: Returns a reference to the Java method METHOD-NAME of
  CLASS-REF with the given PARAMETER-CLASS-REFS.

JMETHOD-LET
  Function: (not documented)

JMETHOD-NAME
  Function: Returns the name of METHOD as a Lisp string

JMETHOD-PARAMS
  Function: Returns a vector of parameter types (Java classes) for METHOD

JMETHOD-RETURN-TYPE
  Function: Returns the result type (Java class) of the METHOD

JNEW
  Function: Invokes the Java constructor CONSTRUCTOR with the arguments ARGS.

JNEW-ARRAY
  Function: Creates a new Java array of type ELEMENT-TYPE, with the given DIMENSIONS.

JNEW-ARRAY-FROM-ARRAY
  Function: Returns a new Java array with base type ELEMENT-TYPE (a string or a class-ref)

JNEW-ARRAY-FROM-LIST
  Function: (not documented)

JNEW-RUNTIME-CLASS
  Function: (not documented)

JNULL-REF-P
  Function: Returns a non-NIL value when the JAVA-OBJECT `object` is `null`,

JOBJECT-CLASS
  Function: Returns the Java class that OBJ belongs to

JOBJECT-LISP-VALUE
  Function: Attempts to coerce JAVA-OBJECT into a Lisp object.

JPROPERTY-VALUE
  Function: (not documented)

JREDEFINE-METHOD
  Function: (not documented)

JREGISTER-HANDLER
  Function: (not documented)

JRESOLVE-METHOD
  Function: Finds the most specific Java method METHOD-NAME on
  INSTANCE applicable to arguments ARGS. Returns NIL if no suitable
  method is found. The algorithm used for resolution is the same used
  by JCALL when it is called with a string as the first parameter
 (METHOD-REF).

JRUN-EXCEPTION-PROTECTED
  Function: Invokes the function CLOSURE and returns the result.
  Signals an error if stack or heap exhaustion occurs.

JRUNTIME-CLASS-EXISTS-P
  Function: (not documented)

JSTATIC
  Function: Invokes the static method METHOD on class CLASS with ARGS.

JSTATIC-RAW
  Function: Invokes the static method METHOD on class CLASS with
  ARGS. Does not attempt to coerce the arguments or result into a Lisp
  object.

MAKE-CLASSLOADER
  Function: (not documented)

MAKE-IMMEDIATE-OBJECT
  Function: Attempts to coerce a given Lisp object into a java-object of the

REGISTER-JAVA-EXCEPTION
  Function: Registers the Java Throwable named by the symbol
  EXCEPTION-NAME as the condition designated by CONDITION-SYMBOL.
  Returns T if successful, NIL if not.

UNREGISTER-JAVA-EXCEPTION
  Function: Unregisters the Java Throwable EXCEPTION-NAME previously registered by REGISTER-JAVA-EXCEPTION.

\end{verbatim}
